\section{Общие положения}

\subsection{Тезаурус}

В данном разделе описаны термины, которые далее будут использоваться для описания технического задания программного обеспечения с наименованием <<Mafia Narration>>

Мобильное приложение --- это программное обеспечение, предназначенное для работы на смартфонах, планшетах и других мобильных устройствах.

Мобильное Android приложение --- это мобильное приложение, предназначенное для работы поверх операционной системы с наименованием <<Android>>.

Мафия подобная игра --- это настольная психологическая карточная игра, оперирующая терминами персонаж и действие персонажа.

Игрок --- это человек, который берёт на себя задачу отыгрывать действия назначенного или выбранного персонажа.

Персонаж --- это действующее лицо в игре, имеющее такие параметры, как: имя, описание, изображение.

Дейсвие персонажа --- это действие заранее прописанное правилами игры, которое выполняет игрок, чтобы достичь поставленной цели, которая прописана в тех же правилах игры.

Сценарий игры --- это строго заданная последовательность действий различных персонажей из игры.

Пул сценариев игры --- это набор правил для создания сценариев игры, в зависимости от конкретных параметров.

\subsection{Назначение и область применения}

Разрабатываемое приложение призвано упростить организацию игрового процесса Мафия подобных игр, путём замены собой роли ведущего. Под ролью ведущего понимается набор действий, таких как: голосовое обозначение этапов игры, проговаривание действий персонажа, которые должен произвести игрок на определённом этапе игры, проговаривание сторонних фраз в определённые промежутки времени для улучшения погружения в игру. Также данное приложение должно уметь работать с такими играми, как: Secret Hitler, серия игр One Night, серия игр Resistance. Все описанные выше возможности должны задаваться пользователем единожды для каждой из игр.
